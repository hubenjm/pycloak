\documentclass[11pt]{amsart}
\usepackage{amssymb, amsthm, amsmath, graphicx}
\usepackage{color}
\usepackage{epsfig}
\usepackage{algorithm, algpseudocode}
\usepackage{epstopdf}

\newtheorem{theorem}{Theorem}
\newtheorem{lemma}{Lemma}
\newtheorem{corollary}[theorem]{Corollary}
\newtheorem{proposition}{Proposition}

\theoremstyle{definition}
\newtheorem{definition}{Definition}

\theoremstyle{definition}
\newtheorem{remark}{Remark}

\theoremstyle{definition}
\newtheorem{example}{Example}

\newcommand{ \hs }[0]{\hspace{.2cm}}
\newcommand{ \vs }[0]{\vspace{.5cm}}
\newcommand{ \rn }[1]{\mathbb{R}^{#1}}

\newcommand{\sign}{\mathop{\mathrm{sign}}}

\newcommand{\C}{\mathbb{C}}
\newcommand{\R}{\mathbb{R}}

\newcommand{\Q}{\mathbb{Q}}
\newcommand{\Z}{\mathbb{Z}}
\newcommand{\N}{\mathbb{N}}
\newcommand{\Hp}{\mathbb{H}}
\newcommand{\Zplus}{\mathbb{Z}^{+}}
\newcommand{\Sn}{\mathbb{S}}
\newcommand{\St}{\mathbb{S}^2}

\newcommand{\floor}[1]{\left\lfoor#1\right\rfloor}
\newcommand{\ceiling}[1]{\left\lceil#1\right\rceil}

\newcommand{\norm}[1]{\left\Vert#1\right\Vert}
\newcommand{\abs}[1]{\left\vert#1\right\vert}
\newcommand{\set}[1]{\left\{#1\right\}}
\newcommand{\innp}[2]{\langle #1,#2\rangle}
\newcommand{\paren}[1]{\left ( #1\right )}
\newcommand{\bvec}[1]{\mathbf{#1}}
\newcommand{\sgn}{\textrm{sgn }}

\newcommand{\eps}{\varepsilon}
\newcommand{\To}{\longrightarrow}
\newcommand{\BX}{\mathbf{B}(X)}

\newcommand{\Dof}{\mathbf{D}}
\newcommand{\D}{\partial}
\newcommand{\grad}{\nabla}
\newcommand{\curl}{\nabla \times}
\renewcommand{\div}{\nabla \cdot}
\newcommand{\ddx}{\frac{\D}{\D x}}
\newcommand{\ddy}{\frac{\D}{\D y}}
\newcommand{\ddz}{\frac{\D}{\D z}}
\newcommand{\ddu}{\frac{\D}{\D u}}
\newcommand{\ddv}{\frac{\D}{\D v}}
\newcommand{\ddt}{\frac{\D}{\D t}}

\newcommand{\bs}{\backslash}

\newcommand{\Log}{\textrm{Log}}
\newcommand{\Arg}{\textrm{Arg}}
\newcommand{\Tan}{\textrm{Tan}}
\newcommand{\Res}{\textrm{Res}}

\newcommand{\im}{\textrm{Im }}
\newcommand{\re}{\textrm{Re}}
\newcommand{\rank}{\text{rank }}
\newcommand{\img}{\text{Im }}
\newcommand{\Ker}{\text{Ker }}
\newcommand{\Id}{\textrm{Id}}
\newcommand{\tr}{\mathrm{tr}}

\newcommand{\T}[1]{\widetilde{#1}}
\newcommand{\wtilde}[1]{\widetilde{#1}}
\newcommand{\wt}[1]{\widetilde{#1}}

\newcommand{\supp}{\textrm{supp}}

\def\ii{{\rm i}}
\def\dd{{\rm d}}
\def\Im{{\it Im}}
\def\ca{{\cal A}}
\def\ct{{\cal T}}
\def\Ba{{\bf a}}
\def\Bb{{\bf b}}
\def\Bc{{\bf c}}
\def\Bd{{\bf d}}
\def\Be{{\bf e}}
\def\Bf{{\bf f}}
\def\Bg{{\bf g}}
\def\Bh{{\bf h}}
\def\Bi{{\bf i}}
\def\Bj{{\bf j}}
\def\Bk{{\bf k}}
\def\Bl{{\bf l}}
\def\Bm{{\bf m}}
\def\Bn{{\bf n}}
\def\Bo{{\bf o}}
\def\Bp{{\bf p}}
\def\Bq{{\bf q}}
\def\Br{{\bf r}}
\def\Bs{{\bf s}}
\def\Bt{{\bf t}}
\def\Bu{{\bf u}}
\def\Bv{{\bf v}}
\def\Bw{{\bf w}}
\def\Bx{{\bf x}}
\def\By{{\bf y}}
\def\Bz{{\bf z}}
\def\BA{{\bf A}}
\def\BB{{\bf B}}
\def\BC{{\bf C}}
\def\BD{{\bf D}}
\def\BE{{\bf E}}
\def\BF{{\bf F}}
\def\BG{{\bf G}}
\def\BH{{\bf H}}
\def\BI{{\bf I}}
\def\BJ{{\bf J}}
\def\BK{{\bf K}}
\def\BL{{\bf L}}
\def\BM{{\bf M}}
\def\BN{{\bf N}}
\def\BO{{\bf O}}
\def\BP{{\bf P}}
\def\BQ{{\bf Q}}
\def\BR{{\bf R}}
\def\BS{{\bf S}}
\def\BT{{\bf T}}
\def\BU{{\bf U}}
\def\BV{{\bf V}}
\def\BW{{\bf W}}
\def\BX{{\bf X}}
\def\BY{{\bf Y}}
\def\BZ{{\bf Z}}
\def\B0{{\bf 0}}

% Abbreviate definitions of bold greek symbols: note that defs.tex is
% needed first
\newcommand{\BGa}{\bfm\alpha}
\newcommand{\BGb}{\bfm\beta}
\newcommand{\BGd}{\bfm\delta}
\newcommand{\BGe}{\bfm\epsilon}
\newcommand{\BGve}{\bfm\varepsilon}
\newcommand{\BGf}{\bfm\phi}
\newcommand{\BGvf}{\bfm\varphi}
\newcommand{\BGg}{\bfm\gamma}
\newcommand{\BGc}{\bfm\chi}
\newcommand{\BGi}{\bfm\iota}
\newcommand{\BGk}{\bfm\kappa}
\newcommand{\BGl}{\bfm\lambda}
\newcommand{\BGn}{\bfm\eta}
\newcommand{\BGm}{\bfm\mu}
\newcommand{\BGv}{\bfm\nu}
\newcommand{\BGp}{\bfm\pi}
\newcommand{\BGt}{\bfm\tau}
\newcommand{\BGvt}{\bfm\vartheta}
\newcommand{\BGr}{\bfm\rho}
\newcommand{\BGvr}{\bfm\varrho}
\newcommand{\BGs}{\bfm\sigma}
\newcommand{\BGvs}{\bfm\varsigma}
\newcommand{\BGj}{\bfm\tau}
\newcommand{\BGu}{\bfm\upsilon}
\newcommand{\BGo}{\bfm\omega}
\newcommand{\BGx}{\bfm\xi}
\newcommand{\BGy}{\bfm\psi}
\newcommand{\BGz}{\bfm\zeta}
\newcommand{\BGD}{\bfm\Delta}
\newcommand{\BGF}{\bfm\Phi}
\newcommand{\BGG}{\bfm\Gamma}
\newcommand{\BGL}{\bfm\Lambda}
\newcommand{\BGP}{\bfm\Pi}
\newcommand{\BGT}{\bfm\tau}
\newcommand{\BGS}{\bfm\Sigma}
\newcommand{\BGU}{\bfm\Upsilon}
\newcommand{\BGO}{\bfm\Omega}
\newcommand{\BGX}{\bfm\Xi}
\newcommand{\BGY}{\bfm\Psi}

\title{PyCloak - Mathematical Background}
\date{}
\author[Mark Hubenthal]{Mark Hubenthal}
\address{Department of Mathematics, University of Houston}
\email{hubenjm@math.uh.edu}

\begin{document}
\maketitle

\section{Background}
\label{sec:background}
\begin{figure}
\centering
\def \svgwidth{0.4\linewidth}
\input{mainsetup.pdf_tex}
\caption{An antenna defined by $\partial D_{a}$ with a control region $D_{c}$ and far field region $B_R(\B0)$ }\label{fig:mainsetup}
\end{figure}
This python module serves the purpose of solving the following general problem. We want to solve the Helmholtz equation
\begin{equation*}
\Delta u - k^{2} u = 0
\end{equation*}
in $\mathbb{R}^{2}$ for $k > 0$, but in such a way where we can cancel some incoming solution $u_{0}(\mathbf{x})$ on a particular region $D_{c} \subset \mathbb{R}^{2}$. We present the details as follows.

Let $B_{R} \subset \mathbb{R}^{2}$ be the ball of radius $R > 0$. We assume $\B0\in D_{a} \subset  B_{R}$ is the region inside a single antenna with $C^{2}$ boundary, $\partial D_{a}$. We also let $D_{c} \subset B_{R}$ be the control region, which is assumed to satisfy $\overline{D_{c}} \cap \overline{D_{a}} = \emptyset$ (see Figure \ref{fig:mainsetup}). The numerical simulations in the current work are performed for the two dimensional case but implementation of the three dimensional case is in progress.

Consider the function space
\begin{equation*}
\Xi = L^{2}(\partial D_{c}) \times L^{2}(\partial B_{R}),
\end{equation*}
endowed with the scalar product
\begin{equation}
(\phi,\psi)_{\Xi} = \int_{\partial D_{c}}\phi_{1}(\mathbf{y})\overline{\psi}_{1}(\mathbf{y})\,dS_{\mathbf{y}} + \int_{\partial B_{R}} \phi_{2}(\mathbf{y})\overline{\psi}_{2}(\mathbf{y})\,dS_{\mathbf{y}},
\end{equation}
which is a Hilbert space. Consider $K: L^{2}(\partial D_{a}) \to \Xi$, the double layer potential operator restricted to $\partial D_{c}$ and $\partial B_{R}$, respectively, defined by
\begin{equation}
\label{EQ:K}
K\phi(\Bx,\Bz) =  (K_{1}\phi(\Bx), K_{2}\phi(\Bz)), \quad \phi \in L^{2}(\partial D_{a}),
\end{equation}
where
\begin{align*}
K_1\phi(\Bx) & = \int_{\partial D_{a}}\phi(\By)\frac{\partial
\Phi(\Bx,\By)}{\partial \mathbf{\nu}_{\mathbf{y}}}ds_{\By}, \mbox{ for }\Bx\in
\partial D_c,\nonumber\\
K_2\phi(\Bz) & = \int_{\partial D_{a}}\phi(\By)\frac{\partial
\Phi(\Bz,\By)}{\partial \mathbf{\nu}_{\mathbf{y}}}ds_{\By}, \mbox{ for }\Bz\in
\partial B_R(\B0).
\end{align*}
Here $\Phi(\Bx, \By)$ represents the fundamental solution of the relevant Helmholtz operator, i.e.,
\begin{equation}
\Phi(\Bx,\By) = \left\{\begin{array}{ll}
\frac{e^{ik|\Bx-\By|}}{4\pi|\Bx-\By|}, & \mbox{ for } d=3\vspace{0.15cm}\\
\frac{i}{4}H_0^{(1)}(k|\Bx-\By|), & \mbox{ for } d=2\end{array}\right.
\end{equation}
with $H_{0}^{(1)} = J_{0} + iY_{0}$ representing the Hankel function of first type.

We also introduce the adjoint operator $K^{*}: \Xi \to L^{2}(\partial D_{a})$, which can be shown to satisfy
\begin{equation}
K^{*}\psi(\mathbf{x}) = \int_{\partial D_{c}} \psi_{1}(\mathbf{y}) \overline{\frac{\partial \Phi(\mathbf{y},\mathbf{x})}{\partial \nu_{\mathbf{x}}}}\,dS_{\mathbf{y}} + \int_{\partial B_{R}} \psi_{2}(\mathbf{y}) \overline{\frac{\partial \Phi(\mathbf{y},\mathbf{x})}{\partial \nu_{\mathbf{x}}}}\, dS_{\mathbf{y}}, \quad \mathbf{x} \in \partial D_{a}. \label{eq:dlpotentialadjoint}
\end{equation}
Now consider the following problem: for a fixed wave number $k > 0$ and fixed $0 < \mu \ll 1$, find a function $h \in C(\partial D_{a})$ such that there exists a $u \in C^{2}(\mathbb{R}^{n} \setminus \overline{D_{a}}) \cap C^{1}(\mathbb{R}^{n} \setminus D_{a})$ solving
\begin{equation}
\left\{ \begin{array}{rl}
(\Delta + k^{2}) u(\mathbf{x}) & = 0 \quad \mathbf{x} \in \mathbb{R}^{n} \setminus \overline{D_{a}}\\
u & = h \quad \textrm{ on $\partial D_{a}$}\\
\|u - f_{1}\|_{C(\overline{D}_{c})} & \leq \mu \\
\|u\|_{C(\mathbb{R}^{n} \setminus B_{R}(\mathbf{0}))} & \leq \mu,
\end{array}
\right. \label{eq:inverseproblem}
\end{equation}
where $f_{1}$ is a solution to the Helmholtz equation in a neighborhood of the control region $D_{c}$. This problem is equivalent to: for a fixed wave number $k>0$ and given a function $f=(f_1,0) \in \Xi$ and $\mu > 0$, find a density function $\phi \in C(\partial D_{a})$ such that the residual $\|K\phi - f\|_{\Xi} $ is small, i.e., such that
\begin{equation}
\|K\phi - f\|_{\Xi} \leq \mu \label{eq:controlinequality}
\end{equation} 

Problem \eqref{eq:controlinequality} is in fact a Fredholm integral equation of the first kind, and it has been proved that the bounded and compact operator $K$ is also one-to-one and has a dense (but not closed) range, thus proving the existence of a class of solutions for \eqref{eq:controlinequality}. Given the fact that $K$ is compact and that its range is not closed, problem \eqref{eq:controlinequality} is ill-posed. By using regularization, one can approximate a solution to problem \eqref{eq:controlinequality} with an arbitrary level of accuracy $\mu \ll 1$. There are several methods known in the literature, but we will use the Tikhonov regularization method. This method, when applied to the operator $K:L^2(\partial D_a)\rightarrow \Xi$ proposes a solution $\phi_{\alpha}\in C(\partial D_{a})$ of the form 
\begin{equation}
\label{Tikhonov}
\phi_{\alpha}=(\alpha I+K^*K)^{-1}K^*f, \quad 0<\alpha \ll 1,
\end{equation}
where $\alpha$ (the Tikhonov regularization parameter) is a suitably chosen regularization parameter. It is known that $\Vert K\phi_\alpha-f\Vert_\Xi\rightarrow 0$ as $\alpha\rightarrow 0$, but the optimal choice of $\alpha$ is an essential step in designing a feasible method (e.g., finding a minimal norm solution), and there are various modalities to do this. 

We will use the Morozov discrepancy principle associated to the following weighted residual space:
\begin{equation}
\label{space-Xi'}
\Xi' = L^{2}(\partial D_{c}, \|f_{1}\|^{-2}dS) \times L^{2}(\partial B_{R}, (2\pi R)^{-1}dS).
\end{equation}
The reasoning behind using the weighted residual space $\Xi'$ for the discrepancy functional defined below (as opposed to $\Xi$) is as follows. Due to the asymptotic behavior of $\frac{\partial \Phi(\mathbf{x},\, \mathbf{y})}{\partial \nu_{\mathbf{y}}} = \mathcal{O}(|\mathbf{x}-\mathbf{y}|^{-1/2})$ as $|\mathbf{x}-\mathbf{y}| \to \infty$, we have that given a fixed density $\phi$, $\|K\phi\|_{L^{2}(\partial B_{R})} = \mathcal{O}(1)$ as $R \to \infty$. In other words, using the space $L^{2}(\partial B_{R})$ with the standard surface measure is not really suited to the decay properties of double layer potential solutions, because the decay of the normal derivative $\partial_{\nu}\Phi$ is too weak. Similarly, we use the relative norm
\begin{equation}
\label{eq:F1}
\frac{\|[K\phi]_{1} - f_{1}\|_{L^{2}(\partial D_{c})}}{\|f_{1}\|_{L^{2}(\partial D_{c})}}
\end{equation}
on $\partial D_{c}$ because this is a useful quantity for determining how good the control is, regardless of the norm of $f_{1}$. 

Thus, for $f=(f_1,0)\in \Xi$, we will consider the weighted residual $\|K\phi - f\|_{\Xi'}^{2}$ defined as  
\begin{equation}
\label{disc-fun}
\|K\phi - f\|_{\Xi'}^{2} = \displaystyle\frac{1}{\|f_{1}\|_{L^{2}(\partial D_{c})}^{2}}\| K_1\phi  - f_{1}\|_{L^{2}(\partial D_{c})}^{2} + \frac{1}{2\pi R} \|K_2\phi \|_{L^{2}(\partial B_{R})}^{2},
\end{equation}
and then make use of the Tikhonov regularization with Morozov's discrepancy principle for the unique choice of $\alpha$, i.e. such that
\begin{equation}
\label{Morozov}
\|K\phi_\alpha - f\|_{\Xi'}^{2} =\delta^2,
\end{equation}
with $\displaystyle\delta^2\leq \mu^2\min\left\{\frac{1}{2\|f_{1}\|^2_{L^{2}(\partial D_{c})}}, \frac{1}{4\pi R}\right\}$.
Please note that $\phi_\alpha$ in \eqref{Morozov} is given by the Tikhonov regularization for the operator  $K:L^2(\partial D_a)\rightarrow \Xi$ as described in \eqref{Tikhonov}. 

We will account for noise and measurement errors and will consider \eqref{Morozov} with $f=(f_1,0)\in \Xi$ replaced by 
\begin{equation}
\label{random-f}
f_{\epsilon} = (f_{1} + \epsilon \widehat{\nu}\|f_{1}\|_{L^{2}(\partial D_{c})}, \, 0)\in \Xi,
\end{equation} 
where $\widehat{\nu} \in L^{2}(\partial D_{c})$ is a random perturbation with $\|\widehat{\nu}\| = 1$ and $f_{1} \in L^{2}(\partial D_{c})$ the far field of a far field observer. 

The goal of this program is to numerically compute the minimal norm solution uniquely determined by \eqref{Morozov}, analyze its stability for given noisy data in $\Xi$ and, in the case of data corresponding to a point source, analyze its sensitivity with respect to parameters such as: mutual distances between $D_a$, $D_c$ and $B_R(\B0)$; wave number $k$; and the location of the point source with respect to $B_R(\B0)$.

\section{Implementation Details}
We discretize the integral operator $K$ via the method of moment collocation. First we choose an approximate basis of functions for $L^{2}(\partial D_{a})$. To do this we suppose the domain $D_{a}$ can be parametrized in polar coordinates by points
\begin{equation*}
(s(\tau)\cos{\tau}, s(\tau)\sin{\tau})), \quad \tau \in [0,2\pi]
\end{equation*}
where $s:\mathbb{R} \to \mathbb{R}_{+}$ is a $2\pi$-periodic smooth function. Using these coordinates, any function $\phi$ defined on $\partial D_{a}$ can be realized via the pullback as a function of $\tau$:
\begin{equation*}
\phi(s(\tau)\cos{\tau}, s(\tau)\sin{\tau}).
\end{equation*}

Now let $n_{a} \in \mathbb{N}$ and let $\tau_{j} = \frac{2\pi j}{n_{a}}$, $0 \leq j \leq n_{a}-1$ be $n_{a}$ equally spaced points on the interval $[0,2\pi)$. We then use the exponential basis functions $\{e^{il\tau}\}_{l=0}^{n_{a}-1}$ for $L^{2}([0,2\pi])$ and approximate a given $\phi \in L^{2}(\partial D_{a})$ via interpolation at the points $\{\tau_{j}\}_{j=0}^{n_{a}-1} \subset [0,2\pi]$. Note that
\begin{align}
\int_{\partial D_{a}}\phi(\mathbf{y}) \frac{\partial \Phi}{\partial \nu_{\mathbf{y}}}(\mathbf{x},\mathbf{y})\,dS_{\mathbf{y}} & = \int_{0}^{2\pi}\phi(s(\tau)\cos{\tau},s(\tau)\sin{\tau})\frac{\partial \Phi}{\partial \nu_{y}}(\mathbf{x}, (s(\tau)\cos{\tau},s(\tau)\sin{\tau})) \notag\\
& \quad \cdot \sqrt{s(\tau)^{2} + s'(\tau)^{2}}\,d\tau.
\end{align}
Furthermore, since $\left( s'(\tau)\cos{\tau} - s(\tau)\sin{\tau}, s(\tau)\cos{\tau} + s'(\tau)\sin{\tau}\right)$ is a tangent vector to $\partial D_{a}$, we have that
\begin{align*}
\nu(\mathbf{y}) = \nu(\tau) & = \frac{(s(\tau)\cos{\tau} + s'(\tau)\sin{\tau}, \, s(\tau)\sin{\tau} - s'(\tau)\cos{\tau})}{\sqrt{ s(\tau)^{2} + s'(\tau)^{2}}}\\
& = \frac{s(\tau)\widehat{\tau} - s'(\tau)\widehat{\tau}^{\perp}}{\sqrt{ s(\tau)^{2} + s'(\tau)^{2}}}.
\end{align*}
is the unit outward normal vector to $\partial D_{a}$. It is then straightforward to compute in the case of Helmholtz equation in 2-D that
\begin{align*}
& \quad \frac{\partial \Phi}{\partial \nu_{\mathbf{y}}}(\mathbf{x}, (s(\tau)\cos{\tau},s(\tau)\sin{\tau}))\\
& = \nabla_{y}\Phi(\mathbf{x}, (s(\tau)\cos{\tau}, s(\tau)\sin{\tau})) \cdot \nu(\tau)\\
& = \frac{ik}{4}H_{0}^{(1)'}(k|\mathbf{x} - s(\tau)\widehat{\tau}|)\frac{s(\tau)\widehat{\tau} - \mathbf{x}}{\sqrt{s(\tau)^{2} + |\mathbf{x}|^{2} - 2s(\tau)\mathbf{x}\cdot \widehat{\tau}}} \cdot \frac{s(\tau)\widehat{\tau} - s'(\tau)\widehat{\tau}^{\perp}}{\sqrt{s(\tau)^{2} + s'(\tau)^{2}}}
\end{align*}

Let $n_{c} \in \mathbb{N}$ be the total number of sample points on $\partial D_{c}$. Also let $n_{R}$ be the total number of sample points on $\partial B_{R}$. We write the $2 \times (n_{c} + n_{R})$ matrix of points
\begin{equation*}
\mathbf{X} := [x_{1}, \ldots, x_{n_{c}+n_{R}}],
\end{equation*}
where each $x_{j}$ is a $2$-vector, $\{x_{j}\}_{j=1}^{n_{c}} \subset \partial D_{c}$ and $\{x_{j}\}_{j=n_{c}+1}^{n_{c} + n_{R}} \subset \partial B_{R}$. 

Suppose now that $D_{c}$ is an annular sector defined by $r_{1} \leq r \leq r_{2}$ and $\theta_{1} \leq \theta \leq \theta_{2}$. Then we have
\begin{equation*}
x_{j} = \left\{ \begin{array}{rl}
\left[ \begin{array}{r}
r_{1}\cos\paren{(j-1/2)\Delta \theta_{in}}\\
r_{1}\sin\paren{(j-1/2)\Delta \theta_{out}} \end{array}\right] & 1 \leq j \leq n_{arc_1}\\
(j-n_{arc_1} - 1/2)\Delta t[\cos(\theta_{2}), \, \sin(\theta_{2})]^{T} &  n_{arc_1} + 1 \leq j \leq n_{arc_1} + n_{s}\\
\left[\begin{array}{r} r_{2}\cos\paren{(n_{arc_1}+n_{arc_2}+n_{s}-j+1/2)\Delta \tau_{2}}\\
 r_{2}\sin\paren{(n_{arc_1}+n_{arc_2}+n_{s}-j+1/2)\Delta \tau_{2}} \end{array}\right] & n_{arc_1} + n_{s} + 1 \leq j \leq n_{c} - n_{s}\\
(n_{arc_1}+n_{arc_2} + 2n_{s} - j + 1/2)\Delta t[\cos(\theta_{1}), \,\sin(\theta_{1})]^{T} & n_{c} - n_{s}+ 1 \leq j \leq n_{c} \end{array}\right\}.
\end{equation*}
Here $\Delta \theta_{in} = \frac{\theta_{2} - \theta_{1}}{n_{arc_1}}$, $\Delta \theta_{out} = \frac{\theta_{2} - \theta_{1}}{n_{arc_2}}$, and $\Delta t = \frac{r_{2} - r_{1}}{n_{s}}$, where $n_{s} = \ceiling{\frac{r_{2} - r_{1}}{r_{1}\Delta \theta_{in}}}$. Moreover, $n_{arc_{1}}$ is a chosen positive integer denoting the number of sample points to use on the inner arc of the control region, and from this we determine $ds_1 = r_{1}(\theta_{2} - \theta_{1})/n_{arc_1}$ and $n_{arc_2} = \ceiling{r_{2}(\theta_{2}-\theta_{1})/ds_{1}}$. In this case, $n_{c} = n_{arc_1}+n_{arc_2} + 2n_{s}$. For $z_{j}$ we simply have $z_{j} = [R\cos\paren{\frac{2\pi}{R}(j-1)}, R\sin\paren{\frac{2\pi}{R}(j-1)}]^{T}$, $1 \leq j \leq n_{R}$. See Figure \ref{fig:numericssetup} for details.
\begin{figure}
\centering
\def \svgwidth{0.6\linewidth}
\input{sectorcontrol.pdf_tex}
\caption{Antenna and control region geometry used for numerical experiments.}\label{fig:numericssetup}
\end{figure}

For each $1 \leq j \leq n_{c} + n_{R}$ and each $0 \leq l \leq n_{a}-1$, we compute $K[e^{il\tau}](x_{j})$ via the approximation
\begin{equation*}
\frac{2\pi}{n_{a}}\sum_{m=0}^{n_{a}-1} \frac{\partial \Phi( x_{j}, [s(\tau_{m})\cos\paren{\tau_{m}}, s(\tau_{m})\sin\paren{\tau_{m}}]^{T})}{\partial \nu_{\mathbf{y}}} e^{il \tau_{m}}\sqrt{s(\tau_{m})^{2} + s'(\tau_{m})^{2}}.
\end{equation*}
If we fix $j$ and vary $l$, we see that the above sum is equivalent to computing the discrete fourier transform of the $n_{a}$-vector
\begin{equation}
\mathbf{v}_{j} := \left[\frac{\partial \Phi( x_{j}, [s(\tau_{m})\cos\paren{\tau_{m}}, s(\tau_{m})\sin\paren{\tau_{m} }]^{T})}{\partial \nu_{\mathbf{y}}}\sqrt{s(\tau_{m})^{2} + s'(\tau_{m})^{2}}\right]_{m=0}^{n_{a}-1}, \label{eq:vj}
\end{equation}
which can be computed efficiently using the Fast Fourier Transform algorithm. In particular
\begin{equation*}
[K\{e^{il\tau}\}(x_{j})]_{l=0}^{n_{a}-1} \approx 2\pi \mathsf{FFT}(\mathbf{v}_{j}),
\end{equation*}
where $\mathsf{FFT}$ is defined on $n_{a}$-vectors $\mathbf{v} = [v_{m}]_{m=1}^{n_{a}}$ by
\begin{equation}
\mathsf{FFT}(\mathbf{v})_{j} = \frac{1}{n_{a}}\sum_{m=1}^{n_{a}}v_{m}e^{\frac{2\pi i (m-1)(j-1)}{n_{a}}}.
\label{eq:fft}
\end{equation}
So the matrix representation of $K$ is then the $n_{a} \times (n_{c} + n_{R})$ matrix
\begin{equation}
A := [2\pi \mathsf{FFT}(\mathbf{v}_{1}), \cdots ,2\pi \mathsf{FFT}(\mathbf{v}_{n_{c}+n_{R}})].
\end{equation}
Now, in order to approximately solve the ill-posed problem $K\phi = f$, we attempt to solve the linear system
\begin{align*}
K_{1}\phi(x_{j}) & = f_{1}(x_{j}), \quad 1 \leq j \leq n_{c}\\
K_{2}\phi(x_{j}) & = f_{2}(x_{j}), \quad 1 \leq j \leq n_{R}.
\end{align*}

Since $A$ is computed with respect to the functions $\{e^{il\tau}\}$, we approximate the coefficients of $\phi$ with respect to the given basis:
\begin{align*}
c_{l} & := \frac{2\pi}{n_{a}}\sum_{m=0}^{n_{a}-1}e^{-il\tau_{m}}\phi(s(\tau_{m})\cos(\tau_{m}), s(\tau_{m})\sin(\tau_{m}))\\
& \approx \frac{1}{2\pi}\int_{0}^{2\pi} e^{-il\tau}\phi(s(\tau)\cos(\tau), s(\tau)\sin(\tau))\,d\tau.
\end{align*}
Let
\begin{equation*}
h = [c_{l}]_{l=0}^{n_{a}-1}.
\end{equation*}

We now numerically compute the Tikhonov regularized solution
\begin{equation*}
h_{\alpha} := (A^{*}A + \alpha I)^{-1}A^{*}f,
\end{equation*}
with $\alpha > 0$. The solution $h_{\alpha}$ gives an approximation of the coefficients $c_{l}$ of the desired density $\phi$ with respect to the functions $\{e^{il\tau}\}_{l=0}^{n_{a}-1}$. We obtain the density $\phi_{\alpha}$ corresponding to $h_{\alpha}$ sampled at the angles $\tau_{m}$ on $\partial D_{a}$ by the formula
\begin{equation*}
\phi_{\alpha}(\tau_{m}) := \sum_{l=0}^{n_{a}-1}[h_{\alpha}]_{l}e^{il\tau_{m}}.
\end{equation*}
Note that we have yet to specify how the regularization parameter $\alpha$ is chosen.

We define the discrepancy function $F(\alpha)$ by
\begin{equation}
F(\alpha) = \|K\phi_{\alpha} - f\|_{\Xi'}^{2} - \delta^{2} \label{eq:F}
\end{equation}
where $\delta > 0$ is a fixed error parameter. This function is not globally increasing for every $\alpha > 0$, but it can be experimentally shown to be monotonically increasing for a range of $\alpha$ that includes the optimal regularization parameter with typical domain setups.

Note that if we split the matrix $A$ into two blocks $A_{near}$ ($n_{c}$ by $n_{a}$) and $A_{far}$ ($n_{R}$ by $n_{a}$) so that
\begin{equation*}
A = \left[ \begin{array}{c}
A_{near}\\
A_{far}
\end{array}\right],
\end{equation*}
then $[A\phi]_{1} = A_{near}\phi$, $[A\phi]_{2} = A_{far}\phi$, and $A^{*}A = A_{near}^{*}A_{near} + A_{far}^{*}A_{far}$. In the discretized setting, instead of (\ref{eq:F1}) we take
\begin{equation}
F(\alpha) =\frac{1}{\|f_{1}\|^{2}} \|A_{near}h_{\alpha} - f_{1}\|_{L^{2}(\partial D_{c})}^{2} + \frac{1}{2\pi R}\|A_{far}h_{\alpha} - f_{2}\|_{L^{2}(\partial B_{R})}^{2} - \delta^{2}
\end{equation}
with
\begin{equation}
h_{\alpha} = (A^{*}A + \alpha I)^{-1}A^{*}f = (A_{near}^{*}A_{near} + A_{far}^{*}A_{far} + \alpha I)^{-1}\paren{A_{near}^{*}f_{1} + A_{far}^{*}f_{2}}.
\end{equation}
We compute
\begin{align}
F'(\alpha) & = \frac{-2\alpha}{\|f_{1}\|_{L^{2}(\partial D_{c})}^{2}}\re\paren{\frac{\partial h_{\alpha}}{\partial \alpha}, \, h_{\alpha}} \notag\\
& \quad + \paren{ \frac{1}{\pi R} - \frac{2}{\|f_{1}\|_{L^{2}(\partial D_{c})}^{2}}}\re\paren{\frac{\partial h_{\alpha}}{\partial \alpha}, \, A_{far}^{*}A_{far}h_{\alpha} - A_{far}^{*}f_{2}} \label{eq:dF1}\\
\frac{\partial h_{\alpha}}{\partial \alpha} & = -(A^{*}A + \alpha I)^{-1}h_{\alpha}, \label{eq:dF2}
\end{align}
where $( \cdot, \cdot )$ denotes the $L^{2}$ inner product on $\partial D_{a}$.

The function $f_{1}$ defined on $\partial D_{c}$ will typically be the trace of a plane wave or of the fundamental solution to the Helmholtz equation. Also, in all numerical examples presented herein, we assume $f_{2} \equiv 0$ on $\partial B_{R}$. A plane wave with frequency $k$ and direction $\xi$ ($\|\xi\| = 1$) is given by
\begin{equation}
e^{ik\xi \cdot \mathbf{x}}, \label{eq:planewavesol}
\end{equation}
and a spherical source is represented as
\begin{equation}
\frac{i}{4}H_{0}^{(1)}(k|\mathbf{x} - \mathbf{x}_{0}|), \label{eq:sphericalsource}
\end{equation}
where $\mathbf{x}_{0}$ is the source point (typically outside of $B_{R}$). Both are solutions to $(\Delta + k^{2})u = 0$.

For such an $f_{1}$, there are some quantities in which we will be interested so as to determine the effectiveness of a given density $\phi$ in solving the problem $K\phi =f$. These are: the relative error of $K\phi$ on $\partial D_{c}$; the $L^{2}$ average of the norm of $K\phi$ on $\partial B_{R}$; the relative and absolute stability of $\phi$ when applying a small perturbation to $f_{1}$; the norm of $\phi$ on $\partial D_{a}$, which is directly related to the power of the antenna. In other words, we will measure
\begin{equation}
\frac{\|K_{1} \phi - f_{1}\|_{L^{2}(\partial D_{c})}}{\|f_{1}\|_{L^{2}(\partial D_{c})}}, \quad \frac{1}{\sqrt{2\pi R}} \|K_{2}\phi\|_{L^{2}(\partial B_{R})}, \label{eq:computevar1}
\end{equation}
\begin{equation}
\frac{\|\phi^{\epsilon}- \phi^{0}\|_{L^{2}(\partial D_{a})}}{\|\phi^{0}\|_{L^{2}(\partial D_{a})}}, \quad \|\phi^{\epsilon} - \phi^{0}\|_{L^{2}(\partial D_{a})}, \label{eq:computevar2}
\end{equation}
and
\begin{equation}
\|\phi\|_{L^{2}(\partial D_{a})}, \label{eq:phinorm}
\end{equation}
where $\phi^{\epsilon}$ is the Tikhonov regularized solution to $K\phi = (f_{1}^{\epsilon},0)$ with $\|f_{1} - f_{1}^{\epsilon}\|_{L^{2}(\partial D_{c})} = \epsilon \|f_{1}\|_{L^{2}(\partial D_{c})}$, and $\phi^{0}$ is the solution with unperturbed $f_{1}$. Furthermore, the regularization parameters $\alpha$ used to determine $\phi^{0}$ and $\phi^{\epsilon}$ are chosen via Newton's Method using (\ref{eq:dF1}), (\ref{eq:dF2}) such that 
\begin{align}
\|K\phi^{0} - f\|_{\Xi'} & = \delta \notag\\
\|K\phi^{\epsilon} - f^{\epsilon}\|_{\Xi'} & = \delta.
\end{align}

\subsection{Noise}
When adding noise to the desired data $(f_{1}, 0)$, since we always want the trace on $\partial B_{R}$ to be $0$, we only add noise to $f_{1}$. We choose a random perturbation $\eta \in L^{2}(\partial D_{c})$ and set
\begin{equation}
f_{1}^{\epsilon} = f_{1} + \epsilon \widehat{\eta} \|f_{1}\|_{L^{2}(\partial D_{c})}, \label{eq:noiseterm}
\end{equation}
where $\epsilon > 0$ represents the relative percentage of noise added. In the discrete case, $\eta$ is chosen to be a vector of $n_{c}$ pseudorandom numbers on the interval $(-1,1)$. Furthermore, for reproducibility, whenever generating $\eta$ using a pseudorandom number generator, we always reset the seed to the same value.

\subsection{Algorithm}
In summary, algorithm \ref{alg:regsolve} gives the rough approach we use to solving the control problem.
\begin{algorithm}
\caption{Basic Collocation Method with Tikhonov Regularization Approach}
\label{alg:regsolve}
\begin{algorithmic}[]
\Require{$k > 0$, $\textrm{tol} > 0$, $\delta > 0$, $\alpha_{i} > \alpha_{min} > 0$, $f = (f_{1},f_{2}) \in \Xi'$, $s(\tau)$.}
	\State{Choose collocation points $\{y(\tau_{j})\}_{j=1}^{n_{a}} \subset \partial D_{a}$, $\{x_{j}\}_{j=1}^{n_{c} + n_{R}} \subset \partial D_{c} \times \partial B_{R}$.}
	\State{Choose line search step parameter $\beta > 1$.}\\\\

	\Comment{Compute matrix representation $A$ of $K:L^{2}(\partial D_{a}) \to \Xi'$}
	\For{$q=1$ to $n_{c} + n_{R}$}
		\State \begin{align*}
A[j, :] & = \left[ \frac{i}{4}\sum_{l=0}^{n-1}e^{im \tau_{l}}\frac{\partial \Phi_{k}(x_{j}, y(\tau_{l}))}{\partial \nu_{\mathbf{y}}} \sqrt{s(\tau_{l})^{2} + s'(\tau_{l})^{2}}\,\frac{2\pi}{n} \right]_{m=0}^{n_{a}-1}\\
& = 2\pi \mathsf{FFT}(\mathbf{v}_{j}).
\end{align*}
	\EndFor
	\State $\mathbf{f}_{1} = [f_{1}(x_{j})]_{j=1}^{n_{c}}$, $\mathbf{f}_{2} = [f_{2}(x_{j})]_{j=n_{c}+1}^{n_{c}+n_{R}}$, $\mathbf{f} = [\mathbf{f}_{1}; \mathbf{f}_{2}]$
	\State Set $\alpha = \alpha_{i}$
	\State Set $h_{\alpha} \leftarrow (A^{*}A + \alpha I)^{-1}A^{*}\mathbf{f}$
	\While{ $F(\alpha) := \|Ah_{\alpha} - \mathbf{f}\|_{\Xi'}^{2} - \delta^{2} > 0$ \textbf{and} $\alpha \geq \alpha_{min}$ }
		\State $\alpha \leftarrow \alpha/\beta$
		\State Recompute $h_{\alpha}$
	\EndWhile
	\If{ $F(\alpha) \leq 0$ }\\
		\Comment{Use current value of $\alpha$ to start Newton's method}
		\While{ $|F(\alpha)| > \mathrm{tol}$ }
			\State $\partial_{\alpha} h_{\alpha} \leftarrow -(A^{*}A + \alpha I_{n_{c} + n_{R}})^{-1}h_{\alpha}$
			\State $F'(\alpha) \leftarrow \frac{-2 \alpha}{\|f_{1}\|^{2}} \re\paren{ h_{\alpha}, \, \partial_{\alpha}h_{\alpha}}$\\
			\State \qquad \qquad $ + \paren{\frac{1}{\pi R} - \frac{2}{\|f_{1}\|^{2}}}\re\paren{ \partial_{\alpha}h_{\alpha},\, A_{far}^{*}A_{far}h_{\alpha} - A_{far}^{*}f_{2}}$
			\State $\alpha \leftarrow \alpha - \frac{F'(\alpha)}{F(\alpha)}$
		\EndWhile
	\EndIf
	
	\State{Compute $\phi_{\alpha}$ from $h_{\alpha}$.}
\end{algorithmic}
\end{algorithm}

\subsection{Typical Parameters Used for Numerical Experiments\label{subsec:parametersetup}}
Here we describe some of the parameters used for the various numerical experiments presented. In the research paper associated with this work, we usually assume that $\partial D_{a}$ is a circle with radius given by either $a = 0.01$ or $a = 0.1$, and that $\partial D_{c}$ is a sector of an annulus with $\theta_{1} = 3\pi/4$ and $\theta_{2} = 5\pi/4$. For the collocation method, we used $n_{a} = 256$ sample points on $\partial D_{a}$, $n_{\mathrm{inner \, arc}} = 256$ (number of points on inner arc of nearfield control region $\partial D_{c}$), and $n_{R} = 256$ (number of sample points on $\partial B_{R}$). We note that increasing $n_{arc_1}$ or $n_{R}$ will put more emphasis on matching $f$ on $\partial D_{c}$ or $\partial B_{R}$, respectively. The discrepancy parameter $\delta$ used for Tikhonov regularization we chose at either $0.01$ or $0.02$. The number of points used on the other boundary segments of $\partial D_{c}$ are chosen so that the arc length differential is approximately constant. The key variables under consideration are $d = r_{1} - a$ (distance from $\partial D_{c}$ to $\partial D_{a}$), $k$, $\epsilon$ (perturbation parameter for adding noise to $f_{1}$), and $\xi$ (direction of plane wave solution).

\end{document}
